\documentclass{article}
\usepackage[utf8]{inputenc}
\usepackage[italian]{babel}
\usepackage{amsmath} % extra math environments
\usepackage{graphicx} % add figures 
\usepackage{subcaption} % add subfigures
\usepackage{hyperref} % add hypertext links
\usepackage{minted} % add code snippets
\usepackage{siunitx} % SI units formatting

\title{Relazione di Laboratorio 4 - Conducibilità Termica}
\author{Iallorenzi Michele - Walhout Francesco}

% Color setup for hyperref
\hypersetup{
    colorlinks=true,
    linkcolor=cyan,
    urlcolor=blue
    }

\begin{document}
    \maketitle

    \section{Introduzione}
    La \emph{conducibilità termica} è una grandezza fisica che misura la
    rapidità con cui il calore viene trasferito da una determinata sostanza per
    conduzione termica.\\
    Vogliamo misurare la conducibilità termica di alcuni materiali,
    per farlo utilizziamo un apparato sperimentale che consiste in due barre cilindriche
    di alluminio e rame, riscaldate ad un estremità da una resistenza e raffreddate 
    all'altra da dell'acqua corrente.
    Misureremo quindi la temperatura in vari punti dei cilindri e la compareremo con
    quella predetta della teoria, per poi calcolare una misura indiretta della 
    conducibilità termica.
    \subsection{Strumenti utilizzati}
    \begin{itemize}
        \item Due barre cilindriche di metalli diversi
        \item Due resistenze connesse in parallelo ad un alimentatore
        \item Un circuito di acqua corrente
        \item Due termoresistenze connesse ad un computer per l'acquisizione dei dati
    \end{itemize}
    \section{Misure ed Analisi}
    Per facilitare la presa dati, le barre metalliche hanno dei fori sul lato lungo,
    equispaziati tra la fonte calda e quella fredda, in cui inserire le termoresistenze.\\
    Per l'acquisizione dei dati abbiamo utilizzato il programma \href{https://pythonhosted.org/plasduino/index.html}{plasduino},
    che fornisce file di testo contenenti le temperature misurate dai due resistori
    ad intervalli di tempo regolari. Abbiamo quindi inserito le termoresistenze e 
    avviato l'acquisizione dei dati simultaneamente per ciascun foro in modo
    da poter osservare il grafico della temperatura in funzione del tempo,
    oltre che la temperatura finale misurata.
    \section{Elaborazione dei dati}
    \section{Conclusioni}

\end{document}
