\documentclass[a4paper]{article}

\usepackage[utf8]{inputenc}
\usepackage[T1]{fontenc}
\usepackage{textcomp}
\usepackage[italian]{babel}
\usepackage{amsmath, amssymb}

\title{Relazione di Laboratorio 2sd - Rimbalzi}
\author{Walhout Francesco - Iallorenzi Michele}

\begin{document}
    \maketitle

    \section{Introduzione}
    Una pallina elastica lasciata cadere da un altezza $h_0$ completa un certo numero di rimbalzi, ad intervalli di tempo non costanti, prima di fermarsi.
    La sua energia meccanica varia di un fattore $ \gamma<0$ per ogni rimbalzo,
    per questo possiamo dire che l'altezza massima  $h_n$ raggiunta tra il rimbalzo
    $n$ e il rimbalzo $n+1$ è  data dalla formula:
    \begin{equation}
        \label{eq:altezza 1}
        h_n=h_0 \gamma^{n}
    \end{equation}
    Lo scopo dell'esperienza è quello di studiare il comportamento della pallina a partire
    dalle misure degli istanti di tempo dei rimbalzi, e confrontare le misure con i modelli
    matematici che predicono l'andamento delle altezze e la frequenza dei rimbalzi.

    \subsection{Strumenti utilizzati}
    \begin{itemize}
        \item Una pallina elastica
        \item Uno smartphone o dispostivo di registrazione audio
        \item Un metro a nastro
    \end{itemize}

    \section{Misure ed Analisi}
    
    \subsection{Preparazione}
    Vogliamo misurare i tempi analizzando una registrazione audio che tiene traccia dei rimbalzi della pallina,
    quindi è necessario posizionare lo strumento di registrazione per terra vicino
    a dove faremo cadere la pallina e assicurarsi che l'ambiente sia sufficientemente silenzioso.
    Per poter conoscere il valore di $h_0$, abbiamo utilizzato un metro a nastro per
    segnare su un muro un segno ad un altezza conosciuta (nel nostro caso $h_0 = 0.80 [m]$), e abbiamo poi lasciato cadere
    la pallina dall'altezza del segno, assicurandoci di lasciare spazio a sufficienza per evitare
    che la palla si scontri con il muro o con lo strumento di misurazione.
    La registrazione è stata avviata poco prima di lasciar cadere la pallina ed arrestata subito dopo l'ultimo rimbalzo.

    \subsection{Misurazione}
    Osservando il grafico delle ampiezze delle registrazioni fatte, si nota che ciascun rimbalzo
    ha un picco iniziale ben definito e una coda più lunga. 
    Abbiamo scelto di campionare come istante di impatto il picco maggiore all'inizio di ciascun rimbalzo,
    ignorando la coda. È possibile selezionare manualmente le misure andando ad osservare il grafico e selezionando i picchi, ma abbiamo deciso di sviluppare un algoritmo per campionare automaticamente i tempi dei rimbalzi (si veda la funzione da riga x a riga y). L'idea è quella di utilizzare una retta costante (y=c), in modo tale che quando essa incontra un picco, questo venga segnato come rimbalzo, inoltre bisogna escludere anche gli altri picchi che vengono segnati al momento dell'impatto.
    
    Per calcolare l'incertezza di misura abbiamo preso un intervallo che contenesse l'onda solo nei momenti di picco, che testimoniano che in quell'intervallo di tempo c'è stato l'impatto con il suolo.
    (inserire la figura "onda.png")
    \begin{equation}
        \sigma = t_f - t_i = 2.986[s] - 2.983 [s] = 0.003 [s]
    \end{equation}
    
    \subsection{Dati}
    I dati così ottenuti sono riassunti nella seguente tabella:\\
    \begin{table}[h!]
        \centering
        \begin{tabular}{|c|c|c|c|c|}
             \hline
             1 & 2 & 3 & 4 & 5 \\ [0.5ex]
             \hline\hline
             () & () & () & () & () \\
             () & () & () & () & () \\
             () & () & () & () & () \\ [1ex]
             \hline
        \end{tabular}
        \caption{Dati sperimentali}
        \label{tab:dati}
    \end{table}
    
    \subsection{Elaborazione dei dati}
    Utilizzando le leggi della cinematica per corpi in moto uniformemente accelerato, ed ignorando quindi l'attrito dell'aria e altre perturbazioni, si possono studiare le altezze dei rimbalzi mediante la formula:
    \begin{equation}
        \label{eq:altezza 2}
        h_n=\frac{1}{8}g\left( t_n-t_{n-1} \right) 
    \end{equation}
    che ci permette di calcolare le altezze a partire dai dati di cui disponiamo, ovvero l'altezza di partenza $h_0 = 0.8[m]$.
    Questa formula è stata ottenuta attraverso questi calcoli:
    \begin{gather*}
        \label{eq:calcoli}
        \begin{cases}
          x(t) = vt - \frac{1}{2} g t^2 \\
          v(t) = v - gt
        \end{cases} \\
        t* = \frac{t_n - t_{n-1}}{2} \\
        x(t*) = \frac{1}{2} g (t*)^2 = \frac{1}{8} g (t_n - t_{n-1})^2
    \end{gather*}
    
    Facendo gli opportuni calcoli, è possibile notare che le misure delle altezze si dispongono secondo una curva esponenziale in un grafico spazio-tempo, il che soddisfa le nostre aspettative considerando l'equazione \ref{eq:altezza 1}, che è proprio un esponenziale.
    (inserire grafico esponenziale)
    Finiti i calcoli, passiamo al codice. Il codice è stato sviluppato in modo tale che prendesse in input una traccia audio e con essa ricavi l'array dei tempi che corrispondo agli omonimi rimbalzi, utilizzando anche l'algoritmo descritto prima nel paragrafo della "Misurazione". Inoltre potremo ottenere il grafico di curve-fit che ci serve
    
    \section{Conclusioni}
    Abbiamo riportato due grafici, dove nel primo (a sinistra) sono state riportate tutte le misurazioni dell'esperienza, mentre nel secondo grafico (a destra) sono state selezionate
    
    Dal grafico che abbiamo riportato qua sotto, possiamo notare che le misure prese dall'esperienza riesco a fare un buon fit della curva esponenziale, infatti osservando il chi-quadro, ...
\end{document}
