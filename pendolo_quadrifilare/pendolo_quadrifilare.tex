\documentclass{article}
\usepackage[utf8]{inputenc}

\title{Relazione di Laboratorio 1 - Pendolo Quadrifilare}
\author{Iallorenzi Michele - Wallout Francesco}

\begin{document}

\maketitle

\section{Introduzione}
    L'esperimento verte sullo studio del moto del pendolo quadrifilare. Questo pendolo è composto da x fili che sorreggono un pezzo di legno. Questi servono per mantenere il pendolo stabile, in modo che faccia le sue oscillazioni su un piano. Inoltre possiamo ...
    Lo scopo dell'esperienza è studiare i moti del pendolo quando non siamo nel regime delle piccole oscillazioni, cercando di capire come varia il periodo del pendolo in funzione dell'angolo di oscillazione.
    
\section{Esperienza}

    \subsection{Strumenti}
    \begin{itemize}
        \item Pendolo Quadrifilare
        \item Cronometro
        \item Fotocellula (per contare le oscillazioni e la velocità istantanea)
    \end{itemize}
    
    \subsection{Misurazione}
    Posizionare il grave ad un ampiezza abbastanza grande per permettere al pendolo di oscillare il più a lungo possibile, facendo selle oscillazioni uniformi. Appena parte il moto del pendolo, bisogna far partire il cronometro ed aspettare che la "bandiera" del pendolo (che serve a registrare la velocità ed il numero delle oscillazioni) non registri più nessun oscillazione.
    
    \subsection{Calcoli}
    Per calcolare l'ampiezza delle oscillazioni, assumiamo trascurabile l'attrito con l'aria e ricaviamoci la relazione energetica fondamentale:
    \begin{equation}
        mgl(1-cos\theta_0) = \frac{1}{2}m v_0^2
    \end{equation}
    ovvero che l'energia potenziale nel punto di massima altezza è la stessa energia (cinetica) nel punto di massima velocità. Si ricava che:
    \begin{equation}
        \theta_0 = arccos(1 -\frac{v_0^2}{2gl})
    \end{equation}.
    Una volta ottenuto l'angolo, bisogna inserirlo nell'equazione del periodo del pendolo, che si può sviluppare secondo un espansione del seno:
    \begin{equation}\label{T}
        T = 2 \pi \sqrt{\frac{l}{g}}(1 + \frac{1}{16}\theta_0^2 + \frac{11}{3072}\theta_0^2
    \end{equation}.
    Nel codice abbiamo inserito solamente il primo sviluppo ($\frac{1}{16}\theta_0^2$), cercando di fare un fit con una "sorta di parabola" (un po' più schiacciata rispetto alla classica $y=x^2$), ma quanti sviluppi ci servono per fare il fit e quanti se ne possono studiare? Se l'incertezza che viene misurata ad un determinato sviluppo $n$, è più piccola dell'incertezza misurata in quello sviluppo, allora si può dire che è sufficiente studiare la funzione T (nell'equazione \ref{T}) allo sviluppo $n$-esimo.
    ...
    Sappiamo che la velocità di un qualsiasi pendolo reale, diminuisce nel tempo, quindi presenta uno smorzamento esponenziale secondo questa funzione:
    \begin{equation}
        v(t) = v_0(0) e^{-\lambda t}
    \end{equation}.
    Più avanti vedremo come calcolare il tempo di smorzamento $\tau = \frac{1}{\lambda}$
    
\section{Elaborazione dei dati}
    
    \subsection{Grafico}
    Il grafico della funzione velocità, considerando il parametro velocità iniziale, si descrive attraverso una curva esponenziale. Facendo un fit con le misure prese in laboratorio, abbiamo ottenuto questo grafico:
    ...
    Il $\chi^2$ di questo grafico è ...
    
    Il modello del periodo, in funzione dell'angolo di oscillazione, verrebbe una parabola del secondo grado (in accordo con le considerazioni sull'ordine degli sviluppi del seno). Il risultato è:
    ...
    
\section{Conclusioni}

\end{document}
