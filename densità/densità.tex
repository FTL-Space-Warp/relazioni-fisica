\documentclass{article}
\usepackage[utf8]{inputenc}
\usepackage[italian]{babel}
\usepackage{amsmath} % extra math environments
\usepackage{graphicx} % add figures 
\usepackage{subcaption} % add subfigures
\usepackage{hyperref} % add hypertext links
\usepackage{minted} % add code snippets
\usepackage{siunitx} % SI units formatting

\title{Relazione di Laboratorio 1 - Misure di densità}
\author{Iallorenzi Michele - Walhout Francesco}

% Color setup for hyperref
\hypersetup{
    colorlinks=true,
    linkcolor=cyan,
    urlcolor=blue
    }

\begin{document}

    \maketitle

    \section{Introduzione}
    Vogliamo misurare indirettamente le densità di alcuni materiali misurando le
    dimensioni e le masse di alcuni cilindri, parallelepipedi, prismi e sfere fatti 
    di questi materiali.
    Supponendo che questi oggetti abbiano densità $\rho$ uniforme, essa si esprime
    in funzione della massa $m$ e del volume $V$ come:
    \begin{equation}
        \rho = \frac{m}{V}
        \label{eq:densità}
    \end{equation}
    \subsection{Strumenti utilizzati}
    \begin{itemize}
        \item 4 Cilindri, 3 parallelepipedi, un prisma a base esagonale e 5 
            sfere di metalli vari
        \item Un calibro cinquantesimale
        \item Un micrometro con risoluzione di $\SI{0.01}{\mm}$ 
        \item Una bilancia elettronica con risoluzione di $\SI{0.001}{\g}$
    \end{itemize}
    \section{Misure}
    Abbiamo misurato le dimensioni degli oggetti utilizzando dove possibile il micrometro,
    e utilizzando il calibro cinquantesimale per le misure maggiori della massima grandezza
    misurabile dal micrometro. In seguito abbiamo misurato le masse di ogni oggetto mediante
    la bilancia. Nel caso del prisma a base esagonale abbiamo preso come misura del suo
    spessore (ovvero del doppio dell'apotema di base) la media dei tre spessori misurabili
    dalle tre coppie di facce opposte.\\
    I dati così ottenuti sono riassunti nella tabella \ref{tab:dati}.
    \begin{table}[h]
        \centering
        \subfloat[Cilindri]{
        \begin{tabular}{ |c|c|c| } 
            \hline
            Diametro [mm] & Altezza [mm] & Massa [g]\\
            \hline\hline
            11.46	&18.83	&5.712\\
            9.97	&16.68	&10.627\\
            5.98	&19.29	&1.453\\
            9.97	&37.7	&24.789\\
            \hline
        \end{tabular}}\\
        \subfloat[Parallelepipedi]{
        \begin{tabular}{ |c|c|c|c| }
            \hline
            Lato 1 [mm] & Lato 2 [mm] & Lato 3 [mm] & Massa [g]\\
            \hline\hline
            10.07	&17.94	&10.50	&4.861	\\
            8.14	&20.03	&17.52	&7.659	\\
            9.96	&9.98	&41.82	&34.986	 \\
            \hline
        \end{tabular}}\\
        \subfloat[Prisma a base esagonale]{
        \begin{tabular}{ |c|c|c|c|c! } 
            \hline
            Doppi apotemi [mm] & Altezza [mm]& Massa [g]\\
            \hline\hline
            9.96 & 22.75	&16.42\\
            9.96&&\\
            9.94&&\\
            \hline
        \end{tabular}}\\
        \subfloat[Cilindri]{
        \begin{tabular}{ |c|c|c| } 
            \hline
            Diametro [mm] & Altezza [mm] & Massa [g]\\
            \hline\hline
            11.46	&18.83	&5.712\\
            9.97	&16.68	&10.627\\
            5.98	&19.29	&1.453\\
            9.97	&37.7	&24.789\\
            \hline
        \end{tabular}}
        \caption{Misure delle dimensioni.}
        \label{tab:dati}
    \end{table}

    \section{Elaborazione dei dati}
    Per prima cosa è necessario calcolare i valori centrali e le relative incertezze dei
    volumi di ciascun oggetto per fare questo abbiamo usato le seguenti formule:\\
    Per i cilindri:
    Per i parallelepipedi:
    Per il prisma a base esagonale:
    Per le sfere:
    \section{Conclusioni}

\end{document}
